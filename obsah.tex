%=========================================================================
% (c) Michal Bidlo, Bohuslav Křena, 2008

\chapter{Úvod}
Úvod tejto bakalarskej prace bude popisany az nakoniec \footnote{Viz Zasady pisania odbornych prac}.
\chapter{Testovanie softvéru}
V~nasledujúcej kapitole sú stručne zhrnuté teoretické vedomosti o~testovaní softvéru.
Kapitola \ref{sekcia:typy_testovania} popisuje, prečo je testovanie softéru dôležité a~aké rôzne typy
testovania softvéru sa dnes používajú.
V~kapitole \ref{sekcia:testovanie_v_praxi} si popíšeme, ako sa automaticky testuje softvér vo firme Acision a~aké nástroje táto firma používa.
Nakoniec kapitola \ref{sekcia:planovac} popisuje plánovač testov, ktorý firma Acision každodenne používa 
pri testovaní rôznych typov softvéru, ktorý firma vyvíja. 
Cieľom tejto bakalárskej práce je úprava tohto plánovača tak, aby podporoval distribúciu testov na viaceré systémy 
a~tým urýchlil čas potrebný na otestovanie softvéru danou sadou testov. 

\section{Typy testovania softvéru} \label{sekcia:typy_testovania}
Testovanie softvéru je v~dnešnej dobe jedným z~kľúčových faktorov v~IT sfére.
Podľa štandardu IEEE 1059 \cite{Ieee} je testovanie proces analýzy softvéru a~detekovanie rozdielností medzi 
existujúcimi a~požadovanými podmienkami a~vyhodnocovaní vlastností testovaného softvéru.
Viaceré zdroje avšak pojem testovania softvéru definujú inak.
Dokument \cite{Swebok} definuje pojem testovania softvéru ako {\it dynamickú} verifikáciu správania programu voči {\it očakávanému} správaniu programu na {\it konečnej} vzorke
testov, vhodne {\it zvolenej} zo zvyčajne nekonečného množstva možných prípadov použitia.
Pressman v~dokumente \cite{Pressman} prehlásil, že testovanie softvéru je kritickým prvkom zabezpečenia kvality softvéru, ktorý má jedinú primárnu úlohu, a~to násť v~ňom chyby.  

Existujú rôzne spôsoby delenia testovania softvéru. Jedno z~hlavných delení je delenie podľa spôsobu testovania.
Toto rozdelenie vychádza z~toho, či je potrebné k~prevedeniu testu daný softvér spustiť, alebo nie.
\subsection*{Statické testovanie}
Statické testovanie je testovanie ktoré nevyžaduje beh softvéru. Statická analýza sa snaží odhaliť niektoré
programátorské chyby ako napríklad syntaktické chyby, neicializované premenné, nesprávna práca s~pamäťou, delenie nulou, opakované zavretie súboru a~ďalšie. 
Medzi statické testovanie patrí napríklad revízia kódu alebo použitie niektorého nástroja pre statické testovanie, napríklad syntaktický analyzátor, sémantický
analyzátor, analyzátor závislostí, atď. Statické testovanie môže byť manuálne alebo automatické. Tento typ testovania je možný v~ľubovolnej fáze vývoja softvéru.
\subsection*{Dynamické testovanie}
Dynamické testovanie vyžaduje beh testovaného softvéru. Dynamické testovanie môže produkovať výsledky, ktoré nie sú so statickou analýzou možné, alebo 
by boli použitím statického testovania časovo náročné. Tento typ testovania vyžaduje spustiteľnú verziu vyvýjaného softvéru.
\\
\\
Medzi ďalšie delenie patrí delenie testovania podľa spôsobu vykonávania testov:
\subsection*{Manuálne testovanie}
Pri manuálnom testovaní vykonáva test používateľ priamou interakciou s~testovaným produktom. 
Tento typ testovania sa používa pokiaľ test potrebuje ľudské ohodnotenie alebo úsudok.

\subsection*{Automatické testovanie}
Automatické testovanie je prevádzané strojom.
Tento typ testovania sa zavádza väčšinou do rozsiahlych projektov.
Využíva sa pri opakovanom spúsťaní veľkého množstva testov, alebo testov s~veľkými množstvami dát.
Pri automatickom testovaní sa využíva nejaký automatizovaný nástroj, pričom môže ísť o~nástroje 
pre vykonávanie testov alebo o~nástroje pre správu testov.
\\
\\
Trocha odlišne sa pristupuje k~deleniu testov na základe toho, aké znalosti máme o~testovanom produkte.
Môže ísť o:
\subsection*{Testovanie pomocou bielej skrinky}
Tento typ testovania vyžaduje prístup k~zdrojovému kódu softvéru. Na základe znalosti zdrojového kódu sa potom vytvárajú testy.
Testovanie pomocou bielej skrinky však nemusí odhaliť neimplementované časti systému, alebo chýbajúce požiadavky.
\subsection*{Testovanie pomocou čiernej skrinky}
Táto metóda nevyžaduje znalosť zdrojového kódu testovaného softvéru počas vytvárania testov.
Pri návrhu testov sa používa externý pohľad na testovaný softvér. 
Produkt berieme ako čiernu skrinku, do ktorej sa nevieme pozrieť.
O~tejto čiernej skrinke vieme len to, ako sa chová navonok a~ako vyzerá.
Pri tomto type testovania sa zameriavame na vstupy a~výstupy programu, bez znalosti toho, ako je naimplementovaný.
\subsection*{Testovanie pomocou sivej skrinky}
Testovanie pomocou sivej skrinky je forma testovania niekde medzi bielou a~čiernou skrinkou. 
Využívajú sa v~nej limitované vedomosti o~implementácií testovaného softvéru.
Nemáme napríklad k~dispozícií celý zdrojový kód, ale iba dizajn softvéru alebo databázu.
\\
\\
Testovanie softvéru môže byť zvyčajne vykonávané na rôzných úrovniach 
procesu vývoja alebo údržby softvéru. Cieľ testu môže byť rôzny: od jedného
modulu, cez skupinu modulov, alebo celého systému.
Toto delenie je na základe úrovní testovania softvéru a~vyzerá nasledovne:
\subsection*{Unit testy}
Tieto testy verifikujú funkcionalitu softvéru v~častiach, ktoré sú testovateľné oddelene.
Unit testy sú definované presnejšie v~štandarde IEEE1008-87 \cite{Ieee_unit}.
\subsection*{Integračné testovanie}
Integračné testovanie je proces v~ktorom sa verifikuje interakcia medzi viacerými softvérovými komponentami.
\subsection*{Systémové testovanie}
Systémové testovanie sa zaoberá správaním celého systému. Behom systémového testovania sa aplikácia testuje ako celok,
a~preto je toto testovanie vhodné pre neskoršie fázach vývoja.
\subsection*{Akceptačné testovanie}
Testuje správanie systému voči požiadavkám zákazníka. Akceptačné testy overujú to, že je daný softvér 
schopný byť nasadený do ostrej prevádzky, a~typicky sú súčasťou prevzatia softvéru zákazníkom.

Testovanie môže byť cieľené na verifikovanie rôznych vlastností softvéru. Na základe toho, na akú časť systému je 
testovanie zamerané, môžeme testovanie rozdeliť na:
\subsection*{Inštalačné testovanie}
Toto testovanie overuje, či je vytvorený softvér možné nainštalovať
na cieľové prostredie. Na tento typ testovania sa môžeme pozrieť ako na 
systémové testovanie ovplyvnené viacerými faktormi, ako napríklad hardwarové požiadavky
alebo požiadavky na operačný systém. 
\subsection*{Testovanie výkonu}
Tento typ testovanie je špeciálne zameraný na to, že softvér spĺňa stanovené požiadavky 
na výkon, ako napríklad doba odozvy alebo počet vykonaných operácií za čas. 
\subsection*{Regresné testovanie}
Podľa štandardu IEEE610.12 \cite{Ieee_glossary} je regresné testovanie \quotedblbase Selektívne pretestovávanie 
systému alebo komponenty na verifikáciu, že zmeny nespôsobili nechcené efekty a~že systém alebo
komponenta stále spĺňa špefikované požiadavky.\textquotedblleft
Myšlienkou tohoto testovania je overiť to, že nové zmeny do funkcionality systému nezaniesli do systému nové typy chýb.
Bežným spôsobom tohoto testovania je periodické spúšťanie testov vytvorených v~minulosti a~kontrolovanie, či sa zmenilo správanie
systému a~či sa chyby opravené v~minulosti znovu neprejavili.
\subsection*{Testovanie použiteľnosti}
Testovanie použitelnosti overuje aké zložité je pre koncového používateľa pracovať s~daným softvérom.
Toto testovanie overuje taktiež prácu s~dokumentáciou alebo napríklad schopnosť zotavenia sa z~chyby. 

Uvedené delenia testovania a~ich typy patria medzi najzákladnejšie. 
Existuje ešte niekoľko spôsobov delenia testovania softvéru, ktoré v~tejto kapitole nie sú popísané.

\section{Testovanie softvéru v~praxi} \label{sekcia:testovanie_v_praxi}
V~nasledujúcej kapitole si popíšeme, ako môže byť riešené testovanie komerčného softwaru v~praxi.
Firma Acision\footnote{http://www.acision.com/}, pre ktorú je táto bakalárska práca vytváraná, vyvíja niekoľko komerčných produktov
v~sfére messagingu. Pri produktoch je kladený veľký dôraz práve na testovanie.
Jedným z~najväčších produktov ktorý firma vyvíja je produkt {\it Message Controller}
\footnote{http://www.acision.com/services/messaging-infrastructure/message-controller} (ďalej len MCO).
Táto bakalárska práca je zameraná pre potreby tohoto produktu a~jeho derivátov.
MCO je komerčný systém, ktorý je na trhu dostupný už niekoľko rokov, a~preto sú hodnoty uvedené v~tejto 
bakalárskej práci z~oddelenia maintenance, ktoré je zodpovedné za údržbu systému a~opravovania chýb v~ňom vyskytnutých.
Jednotlivé produkty sú každodenne testované regresnou sadou testou, ktorá v~prípade MCO pozostáva z~asi 1500 testov.

Údržba produktu MCO funguje formou kontinuálnej integrácie\footnote{angl. Continuous integration}.
Kontinuálna integrácia je vývoj softwaru, kedy členovia týmu spájanú ich prácu častejšie.
Obvykle každý člen týmu integruje svoju prácu aspoň raz denne, čo vedie k~niekoľkým integráciám za deň.
Každá integrácia je verifikovaná automatickým buildom a~testami, aby sa detekovali chyby čo najskôr \cite{Continuous_integration}.

V~praxi to zjednodušene prebieha tak, že je vývojárovi pridelený tzv. ticket, ktorý slúži na popis
nájdenej chyby v~systéme. Pomocou verzovacieho systému si vezme aktuálnu verziu zdrojových súborov a~začne nájdenú chybu opravovať.
Po tom, čo vývojár chybu opraví, zmeny do zdrojového kódu mu skontroluje iný vývojár.
Ak so zmenami do zdrojového kódu bude súhlasiť, uložia tieto zmeny do verzovacieho systému.
Týmto spôsobom sa zaisťuje kontinuálna integrácia na strane vývojárov.

Paralelne s~týmto vývojom beží na jednom servery program, ktorý monitoruje zmeny vo verzovacom systéme.
Ak sa nájde nejaká zmena v~zdrojových kódoch, spustí sa automatický preklad a~zdrojové kódy spolu s~dokumentáciou
sa nanovo skompilujú. Vytvorí sa spustiteľná verzia nového systému. Ak sa preklad zdrojových kódov nepodarí, vývojár ktorý 
danú chybu spôsobil na stav upozornený a~môže sa čo najrýchlejšie pustiť do opravy spôsobenej chyby kompilácie.

Z~pohľadu testera je situácia podobná. Testerovi je taktiež pridelený ticket, pomocou verzovacieho systému si zoberie aktuálnu verziu
všetkých testov a~začne pracovať na novom teste ktorý overí, že vývojár opravil chybu popísanú v~tickete správne.
Ihneď po dokončení testu zaraďuje tento test pomocou verzovacieho systému do aktuálneho repozitára a~vyžiada si
kontrolu napísaného testu. 

V~noci sa potom vezme najnovšia spustiteľná verzia systému, a~na tomto systéme sa pustí aktuálna verzia
všetkých dostupných testov. Pomocou regresného testovania sa potom overí, či sa zmenou zdrojového kódu nezaniesli do systému nové
chyby. Všetky testy sú plne automatizované a~dynamicky overujú, že testovaný systém splňuje všetky požiadavky.

Týmito postupmi je zaistená kontinuálna integrácia. Implementácií kontinuálnej integrácie 
sa venuje napríklad dokument \cite{Continuous_integration_implementation}. 
Postupnou iteráciou týchto procesov sa dosahuje výsledná kvalita systému. Vačšina testov pristupuje k~testovanému
systému metódou čiernej skrinky, takže tester nepotrebuje znalosť zdrojového kódu, a~môže tak vytvárať testy
ešte pred dokončením samotnej implementácie systému. 



\section{Princíp použitého plánovača testov} \label{sekcia:planovac}
Zoznam pouzitych veci
\begin{itemize}
\item prerekvizita je dadydadydaaa
\item cluster je toto
\item a~este tam mame nieco ine
\end{itemize}

\subparagraph{aa1}
aaa1
\subparagraph{aa2}
aaa2
TOto je \textbf{Tucny text} uprostred  normalneho textu

\chapter{Návrh riešenia}
Táto kapitola sa zaoberá základným návrhom
\section{Možnosti rozdeľovania testov}
\section{Komunikácia medzi jednotlivými plánovačmi testov}
\section{Interpretácia výsledkov}
\section{Riešenie nových typov problémov}


\chapter{Implementácia}
V~tejto kapitole sa budeme venovať 
\section{Použité technológie}
\section{Delenie testov na podmnožiny}
\section{Spúšťanie podprocesov a~komunikácia s~nimi}
\section{Sledovanie aktuálneho stavu testovania}
\section{Zbieranie a~vyhodnocovanie výsledkov}
\section{Optimalizácie rozdeľovania testov}


\chapter{Zhodnotenie dosiahnutých výsledkov}
\section{Prínos použitia paralelného plánovača testov}
\section{Porovnanie jednotlivých optimalizácií}


\chapter{Záver}
\section{Zhrnutie}
\section{Možnosti ďalšieho vývoja}

%=========================================================================
